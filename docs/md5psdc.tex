\documentstyle[12pt]{article}
\begin{document}

\title{On ``Pseudocollisions'' in the MD5 Message-Digest Algorithm}
\author{Burton S. Kaliski Jr. \\
{\tt burt@rsa.com} \and
Matthew J.B. Robshaw \\
{\tt matt@rsa.com} \and
RSA Laboratories \\
100 Marine Parkway \\
Redwood City, CA  94065}
\date{April 23, 1993}

\maketitle

A message-digest algorithm maps a message of arbitrary length to a
``digest'' of fixed length, and has three properties: Computing the
digest is easy, finding a message with a given
digest---``inversion''---is hard, and finding two messages with the
same digest---``collision''---is also hard. Message-digest algorithms
have many applications, including digital signatures and message
authentication.

RSA Data Security's MD5 message-digest algorithm, developed by Ron
Rivest \cite{rfc-md5}, maps a message to a 128-bit message digest.
Computing the digest of a one-megabyte message takes as little as a
second.  While no message-digest algorithm can yet be {\em proved}
secure, MD5 is believed to be at least as good as any other that maps
to a 128-bit digest.  Inversion should take about $2^{128}$
operations, and collision should take about $2^{64}$ operations.  No
one has found a faster approach to inversion or collision.

Recent work by den Boer and Bosselaers \cite{den-boer-md5} presents
a special kind of ``pseudocollision'' in MD5's
internal compression function, which maps
a 512-bit message block $x$ and a
128-bit input state $s$ to a 128-bit output
state. They show how to find a message block $x$
and two related input states $s_1$ and $s_2$ that yield the same
output state: $f(x,s_1)$ = $f(x,s_2)$. Their well-thought approach
exploits structural properties of the collision function to find 
a pseudocollision in about $2^{16}$ operations, much less than one
would expect.

Practical implications of this pseudocollision work to the security of
MD5 are not evident. While a real collision in MD5 implies a
pseudocollision (or a ``pseudo-inversion''), a
pseudocollision need not imply a real collision. Indeed, a real
collision, since it involves two different messages, would almost
always involve {\em different} message blocks $x_1$ and $x_2$ such that
$f(x_1,s_1) = f(x_2,s_2)$, but the pseudocollisions have the same
message blocks. Moreover, the input states $s_1$ and $s_2$ would
generally be unrelated, but the pseudocollisions' input states are
the same except for four bits.  There does not seem to be any way to
extend den Boer and Bosselaers' approach to anything beyond the
special pseudocollisions, a limitation they readily admit.

It is reasonable, therefore, to believe that MD5 remains secure. While den
Boer and Bosselaers have found interesting structural properties in
MD5, the properties seem only to lead to special pseudocollisions
and not anything approaching real collisions. Further research, of
course, will give a better understanding of the strengths of MD5 and
other message-digest algorithms, with the eventual hope that
such algorithms can, in some sense, be proved secure.

\bibliographystyle{plain}
\begin{thebibliography}{1}

\bibitem{den-boer-md5}
Bert den~Boer and Antoon Bosselaers.
\newblock Collisions for the compression function of {MD5}.
\newblock In {\it Advances in Cryptology --- Eurocrypt '93}, 1993.
\newblock Preprint.

\bibitem{rfc-md5}
R.L. Rivest.
\newblock {\it {RFC} 1321: The {MD5 Message-Digest Algorithm}}.
\newblock Internet Activities Board, April 1992.

\end{thebibliography}

\end{document}
 
